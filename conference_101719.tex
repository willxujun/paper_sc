\documentclass[conference]{IEEEtran}
\IEEEoverridecommandlockouts
% The preceding line is only needed to identify funding in the first footnote. If that is unneeded, please comment it out.
\usepackage{cite}
\usepackage{amsmath,amssymb,amsfonts}
\usepackage{algorithmic}
\usepackage{graphicx}
\usepackage{textcomp}
\usepackage{xcolor}
\def\BibTeX{{\rm B\kern-.05em{\sc i\kern-.025em b}\kern-.08em
    T\kern-.1667em\lower.7ex\hbox{E}\kern-.125emX}}
\begin{document}

\title{A New Heuristic for Scalable Quantum(-inspired) Annealing on Practical Quadratic Assignment Problems with Block-structural Properties \\
{\footnotesize \textsuperscript{*}Note: Sub-titles are not captured in Xplore and
should not be used}
}

\author{\IEEEauthorblockN{1\textsuperscript{st} Given Name Surname}
\IEEEauthorblockA{\textit{dept. name of organization (of Aff.)} \\
\textit{name of organization (of Aff.)}\\
City, Country \\
email address or ORCID}
\and
\IEEEauthorblockN{2\textsuperscript{nd} Given Name Surname}
\IEEEauthorblockA{\textit{dept. name of organization (of Aff.)} \\
\textit{name of organization (of Aff.)}\\
City, Country \\
email address or ORCID}
\and
\IEEEauthorblockN{3\textsuperscript{rd} Given Name Surname}
\IEEEauthorblockA{\textit{dept. name of organization (of Aff.)} \\
\textit{name of organization (of Aff.)}\\
City, Country \\
email address or ORCID}
\and
\IEEEauthorblockN{4\textsuperscript{th} Given Name Surname}
\IEEEauthorblockA{\textit{dept. name of organization (of Aff.)} \\
\textit{name of organization (of Aff.)}\\
City, Country \\
email address or ORCID}
\and
\IEEEauthorblockN{5\textsuperscript{th} Given Name Surname}
\IEEEauthorblockA{\textit{dept. name of organization (of Aff.)} \\
\textit{name of organization (of Aff.)}\\
City, Country \\
email address or ORCID}
\and
\IEEEauthorblockN{6\textsuperscript{th} Given Name Surname}
\IEEEauthorblockA{\textit{dept. name of organization (of Aff.)} \\
\textit{name of organization (of Aff.)}\\
City, Country \\
email address or ORCID}
}

\maketitle

\begin{abstract}
As quantum computing gains traction, there is a significant effort in applying it to real-world problems. One of the approaches is to use quantum annealing for the Quadratic Assignment Problem (QAP), which in turn has a wide range of applications. However, practical QAPs are often large because the size of a QAP increases quadratically with respect to the number of "items to be assigned". As a result, state-of-the-art quantum(-inspired) annealers, with their limited number of qubits, lack scalability. This paper proposes a new heuristic that enables quantum(-inspired) annealers to solve QAPs of size linear to the number of qubits, provided a certain block-structural property is satisfied. Such property is frequently observed in practical settings such as warehouse allocation. The heuristic is implemented on Fujitsu Digital Annealer (DA), a dedicated quantum-inspired CMOS device that implements annealing. Experiments are performed on standard QAPLIB datasets and randomly generated block-structural QAP instances, and performance is compared against conventional methods such as software simulated annealing. Results demonstrate that the new heuristic 1) produces solutions of comparable quality on standard QAPs, 2) produces superior solutions for block-structural QAPs as well as for warehouse allocation and 3) DA achieves substantial speedup compared to conventional methods. Therefore, via the new heuristic, quantum(-inspired) annealers can effectively solve block-structural QAPs of large sizes and can be applied to practical problems.
\end{abstract}

\begin{IEEEkeywords}
component, formatting, style, styling, insert
\end{IEEEkeywords}

\section{Introduction}
\label{intro}
Quantum(-inspired) annealer is a relatively new kind of hardware. It works by introducing a problem Hamiltonian, which contains biases and couplings between qubits, to the initial Hamiltonian. The energy stay at minimum throughout the annealing process. At the end of annealing, the system is in the eigenstate of the problem Hamiltonian and the qubits are in their classical states.

A major application of quantum computing in the real world is in quantum annealers. Quantum annealers are known to recognise a class of optimisation problems called Quadratic Unconstrained Binary Optimisation (QUBO), which corresponds naturally to the Ising model of the annealing hardware. A QUBO could then be solved in a physical manner.

The proliferation of application of quantum annealers is attributed to the generality of QUBO. Many theoretical and real-world optimisation problems can be modelled as QUBO. Examples include portfolio optimisation in finance \cite{Rosenberg:2016}, traffic flow management \cite{Neukart:2017}, least square solutions to polynomial mathematical equations \cite{Chang:2019}, isomer search in chemistry \cite{Terry:2020}, and also traditional optimisation problems in computer science such as maximum clique \cite{Naghsh:2019} and minimum multicut \cite{Cruz-Santos:2019}.

A focus of this paper is the Quadratic Assignment Problem (QAP) which is another problem that can be recast into QUBO, and thus is amenable to solving by a quantum annealer. It is a well-known combinatorial optimisation problem with a wide range of applications, such as backboard wiring, statistical analysis, placement of electronic components, etc.
QAP can be visualised as the problem of assigning $n$ facilities to $n$ locations. There is a flow between each pair of facilities and a distance between each pair of locations. The aim is to find an assignment of the facilities to the locations such that the sum of all products of flows and their corresponding distances is minimised, subject to the constraints that each facility is assigned to exactly one location and each location contains exactly one facility. Formally, if flow is given as the matrix $(f_{ij})$ and distance is given as the matrix $(d_{kl})$, then the objective function to minimise is \[\sum_{i=1}^{n}\sum_{j=1}^{n}\sum_{k=1}^{n}\sum_{l=1}^{n}f_{ij}d_{kl}x_{ik}x_{jl}\] subject to \begin{align}
\sum_{k=1}^{n}x_{ik}=1 \qquad \forall 1\leq i \leq n \label{ct1}\\
\sum_{i=1}^{n}x_{ik}=1 \qquad \forall 1\leq k \leq n \label{ct2}
\end{align}
$x_{ik}$ is the binary decision variable representing whether facility $i$ goes to location $k$.

The motivation of solving QAP via QUBO and quantum annealer is speed. QAP is computationally difficult because it is NP-hard. Except for very small instances ($n\le15$) which can be solved with exact algorithms, most practical instances takes too long for exact algorithms and various heuristics are used to produce solutions of decent quality with tolerable computation time. For example, (). However, for larger instances, some heuristic algorithms can also take significant time on classical computers. 

Some of the heuristics listed above are amenable to hardware acceleration. For example, (). The hope is that dedicated hardware such as quantum annealer can run much faster than software while preserving solution quality.

However, quantum annealers cannot yet solve practical QAPs of larger sizes. Whereas QAPs in research literature are often experimental in size, ranging from several tens of items to at most over two hundred items, practical QAPs can be way larger. For instance, a warehouse can have hundreds of items, and the number of decision variables in a derived QUBO is easily in the range of $10^4$.

Moreover, as far as QAP is concerned, quantum annealers face a scalability issue. The number of decision variables in a derived QUBO is $O(n^2)$ with respect to the size of QAP. Meanwhile, modern quantum annealers only support a small number of qubits, with each qubit corresponding to a decision variable. The D-Wave quantum annealer has over physical 2000 qubits, but its \textit{chimera} architecture limits the maximum number of nodes in a fully-connected problem graph to around 64 [] because of the lack of full connectivity. Quantum-inspired technologies are better off but not fully capable. For example, Fujitsu Digital Annealer (DA) is a quantum-inspired CMOS ASIC which implements annealing, and it supports over 8000 variables with full connectivity. This means a QAP of maximum size of nearly 90 variables is natively supported on DA without the need for modification. However, this is still too small to satisfy the high demands of problem size in practical QAPs.

This paper attempts to solve this scalability issue for certain special cases of QAPs. In particular, it deals with QAPs having a block-structure. The set of variables in the distance matrix can be split into intervals such that distance between locations in the same interval is relatively small, and distance between locations in different intervals is larger. A real-life instance of this special QAP is in warehouse management, where locations are organised into columns which correspond with intervals. It is easy to travel within a column, but crossing over to another column will require traversing longer distances.

This paper presents a novel divide-and-conquer heuristic that takes advantage of the said block-structure. Combining a maximum k-cut on frequency matrix with a minimum k-cut on distance matrix, the original QAP can be meaningfully divided into sub-problems and each sub-problem will have a size that is $O(n)$ to the number of items. This makes the overall QAP amenable to solving on a quantum device.

This paper is organised as follows. Section 2 describes different aspects of related work. Section 3 elaborates on QAP and gives a more precise definition of the block-structural property. Section 4 explains the new divide-and-conquer heuristic. Section 5 describes experimental setup and computational results. Section 6 discusses some issues and points to future research. Section 7 concludes.

\section{Related Work}
\subsection{QUBO decomposition}
\label{decomp}
There are a few published techniques aimed at decomposing QUBO, which is one level of abstraction lower than QAP. These techniques almost always revolve around the concept of conditioning, which means fixing a subset of variables (or, nomenclatively, ``conditioning'' them) so that the rest of the variables can be solved on hardware. The condition on the subset of variables is then lifted and the space of the subset explored as an outer loop in order to obtain an overall good solution.

In cut-set conditioning, a cut-set of the problem graph is taken out and the remaining disconnected subgraphs are solved separately. The separate solutions are then considered together with the conditioned cut-set to obtain a final solution. This would work for sparser graphs, but block-structural QAP instances in this paper are fully connected, which means the only cut-set is the whole set. This renders cut-set conditioning unusable.

In large neighbourhood local search, a random feasible solution is taken as the starting point and a small subset of variables solvable on hardware are chosen in each iteration. The subproblem generated by this subset is solved while variables outside of the subset are conditioned. This yields an exploration of a very large neighbourhood, whose end result can be optionally plugged into a complementary software local search that gets to a local minimum. Overall, this approach has two complementary steps: the perturbation/jump step to find a point with a good neighbourhood to work with, and the local search step for finding local optimum within such a neighbourhood. Aside from \textit{qbsolv}, the canonical D-wave implementation \cite{Booth:2017} that uses subsets of variables for perturbation, there is another implementation by Oracle \cite{Mihic:2018} which inverts the ways by solving random subsets of variables for local search and updating randomly chosen variables for perturbation. The approach described in this paper falls into this category, but is a special case without perturbation. A neighbourhood is a-priori determined and subsets of variables are chosen heuristically on a higher abstraction level instead of at random.

\subsection{Theories on specially structured QAP}
An interesting line of research deals with specially structured QAPs. Those are QAPs with block structures and have been studied by Cela [] and shown to be well-solvable when the frequency matrix is anti-Monge and when the distance matrix is a specially-defined block matrix. While the QAP instances in this paper have similarly structured distance matrices, numerical values in this paper are more complex, and property of the frequency matrix being anti-Monge is not satisfied. Nevertheless, inspiration can be drawn from this line of work that motivates further theoretical study of the QAP at hand.

\subsection{Use of Quantum(-inspired) Devices in solving QAP}
Much of the quantum annealer application work cited in section \ref{intro} is experimental in scale and does not deal directly with QAP, much less QAP of large scale. There is only one instance of QAP being solved on quantum device with application in flight gate assignment \cite{Stollenwerk:2019}. Reminiscent of what is mentioned in section \ref{intro}, the QAP instance in this work is too big for DA. The scaling method used in this paper is to randomly divide a graph into disconnected components if it is too large, and solve each component separately. This approach, while scalable, is a brute-force one based on randomness and does not give significant insight as to how a QAP can be meaningfully decomposed.

\section{Problem and Method}
In order to enable quantum annealing of a QAP, the problem must first be converted into a Quadratic Unconstrained Binary Optimisation (QUBO) form[], which effectively means constraints \eqref{ct1} and \eqref{ct2} have to be subsumed as part of the objective function in some way. It is customary to do this by encoding the constraints as quadratic penalty terms which augment the objective function. It can be shown that for QUBOs, the optimal solution to the augmented objective function also minimises the original objective function [].

\section{Experiments}

\section{Conclusion and Future Work}

\section*{Acknowledgment}


\bibliographystyle{IEEEtran}
\bibliography{example}

\end{document}
